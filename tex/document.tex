%-*- coding: UTF-8 -*-
% filename.tex
\documentclass[UTF8]{ctexart}
\usepackage{amsmath,mathtools,lmodern,hyperref}
\title{用到的公式}
\author{haodayizhia}
\date{\today}

\bibliographystyle{plain}

\begin{document}
\renewcommand\theequation{%
\thesection.\arabic{equation}}

\maketitle
\tableofcontents
\section{二体问题}

\subsection{核心公式}
\begin{equation}
	\ddot{\vec{r}}=-\frac{\mu}{r^3}\vec{r}
\end{equation}

\subsection{推导用到的向量公式}

\begin{gather}
\vec{a}\times(\vec{b}\times\vec c) =\vec{b}(\vec{a}\cdot\vec{c})-\vec{c}(\vec{a}\cdot\vec{b})\\
\vec{a}\cdot(\vec{b}\times\vec{c})=(\vec{a}\times\vec{b})\cdot\vec{c}
\end{gather}

\subsection{角动量$\vec{h}$守恒}
\begin{equation}
	\vec{h}=\vec{r}\times\dot{\vec{r}}
\end{equation}

\subsection{近地点方向$\vec{B}=\mu\vec{e}$}
\begin{equation}
	\dot{\vec{r}}\times\vec{h}=\frac{\mu}{r}\vec{r}+\vec{B}
\end{equation}

\subsection{运动轨迹}
\begin{equation}
	r=\frac{h^2/\mu}{1+B/\mu\cos \theta}\\
\end{equation}

\begin{equation}
	r=
	\begin{cases}
		a & e=0,\text{圆}\\
		\frac{p}{1+e\cos\theta}=\frac{a(1-e^2)}{1+e\cos\theta} & 0<e<1,\text{椭圆}\\
		\frac{p}{1+\cos\theta} & e=1,\text{抛物线}\\
		\frac{p}{1\pm e\cos\theta}=\frac{a(1-e^2)}{1\pm e\cos\theta} & 1<e,\text{双曲线}
	\end{cases}
\end{equation}

\subsection{活力公式}
\begin{equation}
	\frac{1}{2}v^2-\frac{\mu}{r}=-\frac{\mu}{2a}
\end{equation}

\subsection{anomaly转换}

$\theta$: True anomaly(真近点角), $E$: Eccentric anomaly(偏近点角), $M$: Mean anomaly(平近点角).

\begin{gather}
	a-r=ae\cos E\\
	M=n(t-\tau)=E-e\sin E
\end{gather}

\subsection{Conversion between rv and classical orbit elements}
\paragraph{$rv$ to $\sigma$(注意$\arccos$)}
\begin{equation}
	\begin{cases}
		e=\frac{|\dot{\vec{r}}\times\vec{h}-\frac{\mu}{r}\vec{r}|}{\mu}\\
		a=\frac{h^2}{\mu(1-e^2)}\\
		i=\arccos{\frac{\vec{h}\cdot(0,0,1)}{h}}\\
		\Omega=\arccos\frac{(0,0,1)\times\vec{h}\cdot(1,0,0)}{|(0,0,1)\times\vec{h}|}\\
		\omega=\arccos\frac{\vec{B}\cdot((0,0,1)\times\vec{h})}{B|(0,0,1)\times\vec{h}|}\\
		\theta=\arccos\frac{\vec{r}\cdot\vec{B}}{rB}
	\end{cases}or
\begin{cases}
	a=-\frac{\mu r}{v^2r-2\mu}\\
	e=\sqrt{1-\frac{h^2}{\mu a}}\\
	i=\arccos{\frac{\vec{h}\cdot(0,0,1)}{h}}\\
	\Omega=\arccos\frac{(0,0,1)\times\vec{h}\cdot(1,0,0)}{|(0,0,1)\times\vec{h}|}\\
	\omega=\arccos\frac{\vec{B}\cdot((0,0,1)\times\vec{h})}{B|(0,0,1)\times\vec{h}|}\\
	\theta=\arccos\frac{\vec{r}\cdot\vec{B}}{rB}
\end{cases}
\end{equation}
\paragraph{$\sigma$ to $rv$}
\begin{equation}
	\begin{cases}
		\vec{r}=A\begin{bmatrix}
			\frac{p}{1+e\cos\theta}\\
			0\\
			0
		\end{bmatrix}\\
		\vec{v}=A\begin{bmatrix}
		\frac{he\sin\theta}{p}\\
		\frac{h(1+e\cos\theta)}{p}\\
		0
		\end{bmatrix}\\
		A=\begin{bmatrix*}[r]
		\cos(-\Omega) & \sin(-\Omega) & \\
		-\sin(-\Omega) & \cos(-\Omega) & \\
		& & 1
		\end{bmatrix*}\begin{bmatrix*}[r]
		1 & &\\
		& \cos(-i) & \sin(-i)\\
		& -\sin(-i) & \cos(-i)
		\end{bmatrix*}\begin{bmatrix*}[r]
		\cos(\omega+\theta) & \sin(\omega+\theta) & \\
		-\sin(\omega+\theta) & \cos(\omega+\theta) &\\
		& & 1
		\end{bmatrix*}
	\end{cases}
\end{equation}
\bibliography{math}
\end{document}